\documentclass[]{article}
\usepackage[utf8]{inputenc}

%opening
\title{DVA313 - Group contract\\Project - Solar energy calculator}
\author{Author: Sebastian Lindgren}

\begin{document}

\maketitle

\section{Group members}
\begin{itemize}
	\item Lukas Hamacek
	\item Aliya Hussain
	\item Charlie Höglund
	\item Jonathan Larsson
	\item Sebastian Lindgren
	\item Avalika Podduturu Reddy
\end{itemize}

\section{Rules and Regulations}
\begin{enumerate}
	\item If you cannot attend a group meeting you must tell the group about this. The earlier the better. If you miss a group meeting the attending group members will proceed as usual but you will have limited influence on the matters discussed in that meeting. If you repeatedly miss meetings the group will have to tell the teachers and measures will have to be taken. 
	\item If you are late to a meeting you should tell the group about this on the message service \textit{slack}. The meeting should not be delayed more than 15 minutes if the group does not agree to do so internally.
	\item All decisions should be done democratically by popular vote. Since the size of the group creates a risk for a 50-50-vote situation the project manager will have a deciding vote in these situations. Before the decision each side have a chance to motivate why their decision is the best for the project.
	\item The work to be done is to be put in a backlog available to all users. Deadlines (where applicable) and responsible members should be assigned for each task. The work should then flow by itself but if necessary it is up to the group members to remind eachother about the work. This could be done on the weekly group meetings or on \textit{slack}.
	\item A group meeting should happen at least once each week. In this the respective group members should briefly tell the rest about the work he/she has done and it's current status. This could also easily be done by referencing to their \textit{github} activity. If any member is stuck on their assigned problem the group will help him/her to solve it. This does not mean that a member can keep letting others do their work. It means that the group will help by discussing or in worst case by two members switching jobs.
\end{enumerate}

\section{Work to be done}

This work definition and specification is a crude first version and a proper requirement definition and specification will be created in the near future.

\subsection{Definition}
We will do a \textbf{Solar energy calculator} for Bengt Stridh. It will be created using web technologies. It will be a tool for determining what investments in solar energy that is suitable depending on some parameters.

\subsection{Specification}

A preliminary specification has been provided by the project provider. The tool should be able to dynamically change as the market changes. It should be able to help the users to understand what they are doing and make a good decision. The user should be able to be given some default values. This should be done with great care, maybe by informing the user that the values are default. The program should also
\begin{itemize}
	\item Save used input values from one session to the other
	\item Switch between Swedish and English language
	\item Compare the calculated values for two sets of input parameters.
\end{itemize}
The output should contain proudction cost, profitability, cash flow. Diagrams with present values, cost shares and income shares. There should also be a possibility to make a report file that could be printed with input and output values.

Since the calculator should be able to easily be kept up to date it should be easy to update guiding texts and default values as well as suggested minimum and maximum values. 

If possible the updates of the system should be made by uploading an updated Excel-file. 


\section{Specialized roles}
The group should divide their work and be prepared to do any kind of task. The requirements, preparations, meetings, design, implementation, report and presentation (everything including its appropriate documentation). All parts are to be done by everyone in the group. 

The specialized roles are some work (perhaps ~3-5\% of the total work hours) that the specialized member will work on that task instead of other work (to not work more than the others).

The specialized roles are:
\begin{itemize}
	\item Sebastian: Client communications
	\item Avalika: Documentation and presentation
	\item Jonathan: Github master
	\item Lukas: Project management 
	\item Aliya: 
	\item Charlie: 
\end{itemize}

\section{Working time and work effort}
This project course is 7.5HP which corresponds to 5 hours every workday for 8 weeks or 200 hours in total for each member. The members can chose for themselves how they divide their 200 hours but the majority should be in the start of the project. In these hours all your individual work and group meeting times are counted. The mandatory meetings with the teacher are also counted. 

\section{Signatures}
By signing this group contract you have agreed to the contents of this contract. Every member should sign before project work start.

\begin{itemize}
	\item Lukas Hamacek: \hspace{1.5cm} \makebox[2.5in]{\hrulefill}
	\item Aliya Hussain: \hspace{1.80cm} \makebox[2.5in]{\hrulefill}
	\item Charlie Höglund: \hspace{1.60cm} \makebox[2.5in]{\hrulefill}
	\item Jonathan Larsson: \hspace{1.25cm} \makebox[2.5in]{\hrulefill}
	\item Sebastian Lindgren: \hspace{1.05cm} \makebox[2.5in]{\hrulefill}
	\item Avalika Podduturu Reddy: \hspace{0.0cm} \makebox[2.5in]{\hrulefill}
\end{itemize}

\end{document}
